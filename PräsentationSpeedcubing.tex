\documentclass{beamer}

% used Commands for installing everything needed + VSCode Packages

% sudo apt install texlive-latex-extra
% sudo apt -y install texlive-lang-german
% sudo apt install chktex
% sudo apt install latexmk


%Style
\mode<presentation>{
    %Theme
    %\usetheme{default}
    %\usetheme{AnnArbor}
    %\usetheme{Antibes}
    %\usetheme{Bergen}
    %\usetheme{Berkeley}
    %\usetheme{Berlin}
    %\usetheme{Boadilla}
    %\usetheme{CambridgeUS}
    %\usetheme{Copenhagen}      ||||||
    %\usetheme{Darmstadt}
    %\usetheme{Dresden}
    %\usetheme{Frankfurt}
    %\usetheme{Goettingen}
    %\usetheme{Hannover}
    %\usetheme{Ilmenau}
    %\usetheme{JuanLesPins}
    %\usetheme{Luebeck}
    %\usetheme{Madrid}
    %\usetheme{Malmoe}
    %\usetheme{Marburg}
    %\usetheme{Montpellier}
    %\usetheme{PaloAlto}
    %\usetheme{Pittsburgh}
    %\usetheme{Rochester}
    %\usetheme{Singapore}
    %\usetheme{Szeged}
    %\usetheme{Warsaw}

    %Farb-Theme
    %\usecolortheme{albatross}
    %\usecolortheme{beaver}
    %\usecolortheme{beetle}
    %\usecolortheme{crane}
    %\usecolortheme{dolphin}      |||||||
    %\usecolortheme{dove}
    %\usecolortheme{fly}
    %\usecolortheme{lily}
    %\usecolortheme{orchid}
    %\usecolortheme{rose}
    %\usecolortheme{seagull}
    %\usecolortheme{seahorse}
    %\usecolortheme{whale}
    %\usecolortheme{wolverine}

    %\setbeamertemplate{footline} % Fußzeile entfernen
    %\setbeamertemplate{footline}[page number] % Fußzeile entfernen und durch Folien-Zahl ersetzen

    %\setbeamertemplate{navigation symbols}{} % Navigations-Symbole entfernen
}

%Packages
\usepackage[utf8]{inputenc} %Deutsche Umlaute
\usepackage[ngerman]{babel} %Deutsche Sprache
\usepackage{graphicx}       %für Bilder
\usepackage{booktabs}       %für \toprule \midrule \bottomrule in Tabellen


%Einstellungen der Präsi
\title{Speedcubing}
\author{Kimi Löffel}
\institute[Bbc]{ICT Berufsbildungscenter}
\date{\today}



\begin{document}

    %Titelseite
    \begin{frame}
        \titlepage{}
        \end{frame}



    %Inhaltsverzeichnis
    \begin{frame}{Inhalt}
        \tableofcontents
    \end{frame}



    \section{What is it?}


        %What is Speedcubing
        \begin{frame}{What is it?}
            \begin{itemize}
                \item Speedsolving a puzzle \pause{}
                \item Collecting Puzzles \pause{}
                \item Modders
            \end{itemize}
        \end{frame}
    


    \section{Puzzles}
        \subsection{WCA}
        
            \subsubsection{What is the WCA?}



                %What is the WCA
                \begin{frame}{What is the WCA}
                    \begin{itemize}
                        \item World Cube Association \pause{}
                        \item Organizes Competitions \pause{}
                        \item Award official records
                    \end{itemize}
                \end{frame}

            \subsubsection{WCA puzzles/events}

                %All WCA puzzles and events
                \begin{frame}{WCA puzzles and events}
                    \begin{columns}[c] 
                        \column{.25\textwidth} %linke spalte
                            \begin{itemize}
                                \item 3x3x3 \pause{}
                                \item 2x2x2 \pause{}
                                \item 4x4x4 \pause{}
                                \item 5x5x5 \pause{}
                                \item 6x6x6 \pause{}
                                \item 7x7x7 \pause{}
                                \item 3x3x3 BLD \pause{}
                                \item 3x3x3 FMC \pause{}
                                \item 3x3x3 OH \pause{}
                            \end{itemize}
                        \column{.25\textwidth} % mittlere Spalte
                            \begin{itemize}
                                \item Clock \pause{}
                                \item Megaminx \pause{}
                                \item Pyraminx \pause{}
                                \item Skewb \pause{}
                                \item Square-1 \pause{}
                                \item 4x4x4 BLD \pause{}
                                \item 5x5x5 BLD \pause{}
                                \item 3x3x3 Multi-BLD \pause{}
                            \end{itemize}
                        \column{.25\textwidth} %rechte spalte 1

                            \begin{figure}
                                \includegraphics[width=0.6\textwidth]{clock.jpg}
                            \end{figure}

                            \begin{figure}
                                \includegraphics[width=0.6\textwidth]{mega.jpg}
                            \end{figure}

                            \begin{figure}
                                \includegraphics[width=0.7\textwidth]{pyra.jpg}
                            \end{figure}

                        \column{.25\textwidth} %rechte spalte 2

                            \begin{figure}
                                \includegraphics[width=0.9\textwidth]{skewb.jpg}
                            \end{figure}

                            \begin{figure}
                                \includegraphics[width=0.8\textwidth]{square1.jpg}
                            \end{figure}
                    \end{columns}
                \end{frame}

        \subsection{non WCA}
            \begin{frame}{non WCA puzzles}
                \begin{figure}
                    \includegraphics[width=\textwidth]{nonWCAall.jpg}
                \end{figure}
            \end{frame}
        \subsection{Shops}
    \section{Solving methods}
        \subsection{Beginners}
        \subsection{CFOP}
        \subsection{ROUX}
        \subsection{others}
    \section{What about me?}
        %Tabellen
        \begin{frame}{Tabelle}
            \begin{table}
                \begin{tabular}{l l l}
                    \toprule
                    \textbf{2015} & \textbf{2016} & \textbf{2017}\\
                    \midrule
                    0.531 & 0.03262 & 0.562 \\
                    0.122 & 0.15681 & 0.910 \\
                    0.653 & 0.09271 & 0.296 \\
                    \bottomrule
                \end{tabular}
                \caption{Verteilung}
            \end{table}
        \end{frame}

        %Formeln
        \begin{frame}{Theorem}
            \begin{theorem}[Formel xy]
                $P_{anda}=mc^2$
            \end{theorem}
        \end{frame}
\end{document}